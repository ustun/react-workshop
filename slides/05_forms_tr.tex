% Created 2015-06-13 Sat 08:26
\documentclass[presentation]{beamer}
\usepackage[utf8]{inputenc}
\usepackage[T1]{fontenc}
\usepackage{fixltx2e}
\usepackage{graphicx}
\usepackage{longtable}
\usepackage{float}
\usepackage{wrapfig}
\usepackage{rotating}
\usepackage[normalem]{ulem}
\usepackage{amsmath}
\usepackage{textcomp}
\usepackage{marvosym}
\usepackage{wasysym}
\usepackage{amssymb}
\usepackage{hyperref}
\tolerance=1000
\usetheme{Rochester}
\author{Ustun Ozgur}
\date{2015-06-13 Cumartesi}
\title{React ile Formlar}
\hypersetup{
  pdfkeywords={},
  pdfsubject={},
  pdfcreator={Emacs 24.4.1 (Org mode 8.2.10)}}
\begin{document}

\maketitle

\begin{frame}[label=sec-1]{Formlar}
\begin{itemize}
\item Kontrolsüz (Uncontrolled)
\item Kontrollü (Controlled)
\end{itemize}
\end{frame}

\begin{frame}[fragile,label=sec-2]{Kontrolsüz (Uncontrolled)}
 \begin{itemize}
\item Önerilmez
\item Ana DOM bileşenleri veriyi saklar, state değil
\item Değerler olaylardan (event) ya da ref'ler üzerinden DOM elemanlarından okunur
\item ref: bir çocuk bir eşsiz bir gösterici (ünique identifier)
\item \texttt{this.refs} aynı \texttt{props}, \texttt{state} gibi ref'lerden oluşan bir nesnedir
\end{itemize}
\end{frame}

\begin{frame}[fragile,label=sec-3]{Refs}
 \begin{block}{Örnek}
\begin{verbatim}
<div>
    <span ref="inner">This is the inner text</span>
</div>
\end{verbatim}

\begin{itemize}
\item İçteki span elemanına şu şekilde ulaşılabilir \texttt{this.refs.inner}

\item Gerçek bir DOM düğümüne ulaşmak için, sunlar kullanılabilir:
\begin{itemize}
\item \texttt{React.findDOMNode}: tercih edilen
\item \texttt{this.refs.inner.getDOMNode}: eski yöntem
\end{itemize}

\item Değeri almak için DOM düğümüne eriştikten sonra value değerini okuyun
\item \texttt{this.refs.inner.getDOMNode().value}
\end{itemize}
\end{block}
\end{frame}


\begin{frame}[fragile,label=sec-4]{Örnek: Kontrolsüz Girdili (İnput) Bir Form}
 \begin{verbatim}
<form onSubmit={this.onSubmit}>

<input ref={name} placeholder="What is your name"/>
<input ref={surname} placeholder="What is your surname?"/>
<button type="submit">Submit</button>

</form>
\end{verbatim}
\end{frame}

\begin{frame}[fragile,label=sec-5]{Örnek (devam)}
 \begin{verbatim}
var MyForm = React.createClass({
render: function () {

  return <form onSubmit={this.onSubmit}>
    <input ref="name" placeholder="What is your name"/>
    <input ref="surname"
     placeholder="What is your surname?"/>
    <textarea ref="comment" placeholder="Comment"/>
    <label><input ref="fromDenmark"
            type="checkbox"/>Are you from Denmark?</label>
    <label>Gender:
    <select ref="gender">
    <option>Male</option>
    <option>Female</option>
    </select></label>
    <button type="submit">Submit</button>

    </form>
\end{verbatim}
\end{frame}


\begin{frame}[fragile,label=sec-6]{Örnek (devam)}
 \begin{verbatim}
onSubmit: function (e) {
  e.preventDefault();
  var name = React.findDOMNode(this.refs.name).value;
  var surname = React
        .findDOMNode(this.refs.surname).value;
  var comment = React
        .findDOMNode(this.refs.comment).value;
  var fromDenmark = React
        .findDOMNode(this.refs.fromDenmark).checked;
  var gender = React.findDOMNode(this.refs.gender).value;

  console.log("Form submitted with",
              name, surname, comment,
              fromDenmark, gender);

}});

React.render(<MyForm/>, document.body);
\end{verbatim}
\end{frame}

\begin{frame}[label=sec-7]{Özet}
\begin{itemize}
\item İşaret kutuları için (checkbox), value yerine checked özelliğini kullanılır
\item textarea bileşeninin çocuk elemanları yoktur, HTML'de vardır. textarea için
yine value değeri kullanılır.
\item defaultValue : varsayılan (ilk )değer için bu kullanılır.
\item Formların bir de onSubmit adında bir olay fonksiyonu vardır, ama bu aynı
zamanda formun asıl metodunu (GET/POST) çalıştırır, bu nedenle
preventDefault kullanarak asıl metodun çağrılması engellenir.
\item Buradaki olaylar (Event) sentetik olay adı verilen React'e özel olaylardır
ve ana DOM olaylarını normalize ederler.
\end{itemize}
\end{frame}

\begin{frame}[label=sec-8]{Kontrollü Formlar}
\begin{itemize}
\item Tercih edilen yöntem
\item Her girdi bileşeninin bir değer ya da işaretlenmiş özelliği vardırk
\item Her girdi bileşeni bir onChange olayı yaratır.
\item State'i bu onChange olayında değiştiririz.
\end{itemize}
\end{frame}

\begin{frame}[fragile,label=sec-9]{Örnek: Kontrollü Bir girdi (ilk değer)}
 \begin{verbatim}
var MyForm = React.createClass({

  getInitialState: function () {
    return {name: ''};
  },
  render: function () {
    return <form onSubmit={this.onSubmit}>
      <input value={this.state.name} placeholder="What is your name?"/>
      </form>
  }
})
\end{verbatim}
\end{frame}

\begin{frame}[fragile,label=sec-10]{Örnek (devam) (olay fonksiyonları)}
 \begin{verbatim}
var MyForm = React.createClass({

  getInitialState: function () {
    return {name: ''};
  },
  changeName: function (e) {
    this.setState({name: e.target.value});
  },
  render: function () {
    return <form onSubmit={this.onSubmit}>
      <input onChange={this.changeName}
       value={this.state.name}
       placeholder="What is your name?"/>
      </form>}})
\end{verbatim}
\end{frame}

\begin{frame}[fragile,label=sec-11]{Örnek (submit (gönder) metodu)}
 \begin{verbatim}
var MyForm = React.createClass({
  onSubmit: function () {
    console.log("the form values are", this.state.name); },
  getInitialState: function () {
    return {name: ''};  },
  changeName: function (e) {
    this.setState({name: e.target.value});  },
  render: function () {
    return <form onSubmit={this.onSubmit}>
      <input onChange={this.changeName}
       value={this.state.name}
       placeholder="What is your name?"/>
      </form>  }})
\end{verbatim}
\end{frame}


\begin{frame}[label=sec-12]{Formlara Yönelik Diğer Olaylar}
\begin{itemize}
\item onBlur
\item onFocus
\end{itemize}
\end{frame}

\begin{frame}[fragile,label=sec-13]{Hata Ayıklama (Debugging) için bir İpucu}
 \begin{itemize}
\item State'i görsel olarak öldüğü gibi ekrana yazdırma
\item \texttt{JSON.stringify(this.state, null, 4)} kullanarak state'i yazdırabiliriz.
\end{itemize}
\end{frame}

\begin{frame}[fragile,label=sec-14]{Örnek:}
 \begin{verbatim}
var MyForm = React.createClass({
  onSubmit: function () {
    console.log("the form values are", this.state.name); },
  getInitialState: function () {
    return {name: ''};  },
  changeName: function (e) {
    this.setState({name: e.target.value});  },
  render: function () {
    return <form onSubmit={this.onSubmit}>
      <input onChange={this.changeName}
       value={this.state.name}
       placeholder="What is your name?"/>
      <pre>{JSON.stringify(this.state, null, 4)}</pre>
      </form>  }})
\end{verbatim}


DEMO
\end{frame}

\begin{frame}[fragile,label=sec-15]{Alıştırma 1/3:}
 \begin{itemize}
\item Kullanıcının sadece sayı girebildiği bir kredi kartı form input'u ekleyin.
\end{itemize}

İpucu: Bir değişkenin sadece sayılardan oluştuğunu anlamak için su regex'i kullanabilirsiniz \texttt{/\textasciicircum{}\textbackslash{}d+\$/.test(foo)}

\begin{itemize}
\item Birkaç sayı girdikten sonra tamamen silebiliyor musunuz? Silemiyorsanız neyi
değiştirmeniz gerekli? Neden? İpucu: Ya regex'i değiştirin ya da başka bir
yöntem bulun.

\item Tireleri girecek şekilde regex'i değiştirin.
\end{itemize}
\end{frame}

\begin{frame}[fragile,label=sec-16]{Alıştırma 2/3:}
 \begin{itemize}
\item Kullanıcı ad alanına giriş yaparken onları cinsiyetlerine göre Bey/Hanım,
olarak selamlayın. Örneğin adını Ali olarak girerse Merhaba Ali Bey desin,
Ayşe olarak girerse Merhaba Ayşe Hanım desin. Erkek adları: \texttt{["Ali",
  "Ahmet"]}. Kadın adları \texttt{["Ayşe", "Fatma"]}.
\item Bunun dışındaki adlara sadece adıyla hitap edin. Merhaba Hüseyin.
\item Eğer ad yoksa, ekranda Merhaba yazmasın, ekran boş olsun.
\end{itemize}
\end{frame}

\begin{frame}[label=sec-17]{Alıştırma 3/3:}
\begin{itemize}
\item Kullanıcı ad alanından kart alanına geçerken eğer 3 harften kısa bir ad
yazdıysa bir uyarı oluşturun. Ancak daha alanı terk etmediyse bu uyarı
görüntülenmemeli.
\end{itemize}

İpucu: Takip edilmesi gereken state değişkenleri hakkında düşünün. Bir
değişkenin geçerli olup olmadığı bilgisi state'te tutulmalı mı? Artıları ve
eksikleri düşünün.

\begin{itemize}
\item Submit tuşuna tıklayınca bir validasyon ekleyin. İsim en az üç harf, kredi
kartı tireler hariç en az 16 rakam olmalı. Formun validasyon değerini
state'te tutmalı mıyız? Bu konuya kafa yörün.

\item Todo uygulamasında, yeni bir todo ekleme özelliğini ekleyin.
\end{itemize}
\end{frame}
% Emacs 24.4.1 (Org mode 8.2.10)
\end{document}
