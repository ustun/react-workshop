% Created 2015-11-03 Tue 05:40
\documentclass[presentation]{beamer}
\usepackage[utf8]{inputenc}
\usepackage[T1]{fontenc}
\usepackage{fixltx2e}
\usepackage{graphicx}
\usepackage{grffile}
\usepackage{longtable}
\usepackage{wrapfig}
\usepackage{rotating}
\usepackage[normalem]{ulem}
\usepackage{amsmath}
\usepackage{textcomp}
\usepackage{amssymb}
\usepackage{capt-of}
\usepackage{hyperref}
\usetheme{default}
\usecolortheme{beaver}
\author{Ustun Ozgur}
\date{2015-11-03 Tue}
\title{State}
\hypersetup{
 pdfauthor={Ustun Ozgur},
 pdftitle={State},
 pdfkeywords={},
 pdfsubject={},
 pdfcreator={Emacs 24.4.1 (Org mode 8.3.2)},
 pdflang={English}}
\begin{document}

\maketitle


\begin{frame}[fragile,label={sec:orgheadline1}]{State}
 \begin{itemize}
\item Handling apps with changing state
\item Two important methods: \texttt{getInitialState} and \texttt{setState}
\item \texttt{setState}: merges the input to current state
\item Also \texttt{replaceState}, replaces the state with the input
\item state can be accessed from render as \texttt{this.state}
\end{itemize}
\end{frame}

\begin{frame}[fragile,label={sec:orgheadline2}]{Example: Counter}
 \begin{verbatim}
var Counter = React.createClass({

  getInitialState: function () {
    return {counter: 0};
  },
  increment: function () {
    this.setState({counter: this.state.counter + 1});
  },
  render: function () {
    return <div onClick={this.increment}>
      Counter value is {this.state.counter}
    </div>;

  }

});
\end{verbatim}
\end{frame}

\begin{frame}[label={sec:orgheadline3}]{Overview}
\begin{itemize}
\item We have `onClick` handler that calls `setState` with state incremented
by 1.
\item This is the essence of state management with React.
\end{itemize}
\end{frame}

\begin{frame}[label={sec:orgheadline4}]{Passing State as Props to Children}
\begin{itemize}
\item An owner component can pass its state as props to its children
\item One component's state is another's props.
\end{itemize}
\end{frame}

\begin{frame}[fragile,label={sec:orgheadline5}]{Example: Counter in Counters}
 \begin{verbatim}
var Counter = React.createClass({
  render: function () {
    return <div>
      Counter value is {this.props.counter}
      <button onClick={this.props.click}>Increment</button>
      </div>
  }

});
\end{verbatim}
\end{frame}

\begin{frame}[fragile,label={sec:orgheadline6}]{Example: Counters Component}
 \begin{verbatim}
var Counters = React.createClass({
  getInitialState: function () {
    return {counter1: 10, counter2: 0};  },
  incrementCounter1: function () {
    this.setState({counter1: this.state.counter1 + 1,
                   counter2: this.state.counter2 - 1})  },
  incrementCounter2: function () {
    this.setState({counter1: this.state.counter1 - 1,
                   counter2: this.state.counter2 + 1})  },
  render: function () {
    return <div>
      <Counter click={this.incrementCounter1}
       counter={this.state.counter1}/>
      <Counter click={this.incrementCounter2}
       counter={this.state.counter2}/>
      </div>  }})
\end{verbatim}
\end{frame}

\begin{frame}[fragile,label={sec:orgheadline7}]{A More Succint Version using bind}
 \begin{verbatim}
var Counters = React.createClass({
  getInitialState: function () {
    return {counter1: 10, counter2: 0};    },
  incrementCounter: function (increment) {
    this.setState({
      counter1: this.state.counter1 + increment,
      counter2: this.state.counter2 - increment})    },
  render: function () {
    return <div>
      <Counter
       click={this.incrementCounter.bind(this, 1)}
       counter={this.state.counter1}/>
      <Counter
       click={this.incrementCounter.bind(this, -1)}
       counter={this.state.counter2}/>
      </div>    }})
\end{verbatim}
\end{frame}


\begin{frame}[label={sec:orgheadline8}]{Summary}
\begin{itemize}
\item Child components never update their props directly.
\item They inform the parent component that some event has happened via functions
passed as props.
\item The parent modifies its state.
\item React handles propagating the state as props downward.
\item Everything is re-rendered.
\end{itemize}
\end{frame}

\begin{frame}[label={sec:orgheadline9}]{Exercise:}
\begin{itemize}
\item Modify the todo app so that the list items are stored in a store variable
instead of being passed as props.
\item Implement toggling of completed status of todo items.
\item Hint: Should the todo items be a list of strings as it was previously? Is a
list of objects better?
\item Implement filtering the todo list by completed items.
\item Hint: Store whether the filter is in effect by saving it in a state
variable.
\item Implement removing a todo item.
\item Implement showing the number of remaining items.
\end{itemize}
\end{frame}
\end{document}
