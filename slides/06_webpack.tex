% Created 2015-11-03 Tue 05:45
\documentclass[presentation]{beamer}
\usepackage[utf8]{inputenc}
\usepackage[T1]{fontenc}
\usepackage{fixltx2e}
\usepackage{graphicx}
\usepackage{grffile}
\usepackage{longtable}
\usepackage{wrapfig}
\usepackage{rotating}
\usepackage[normalem]{ulem}
\usepackage{amsmath}
\usepackage{textcomp}
\usepackage{amssymb}
\usepackage{capt-of}
\usepackage{hyperref}
\usetheme{default}
\usecolortheme{beaver}
\author{Ustun Ozgur}
\date{2015-11-03 Tue}
\title{Webpack and React}
\hypersetup{
 pdfauthor={Ustun Ozgur},
 pdftitle={Webpack and React},
 pdfkeywords={},
 pdfsubject={},
 pdfcreator={Emacs 24.4.1 (Org mode 8.3.2)},
 pdflang={English}}
\begin{document}

\maketitle

\begin{frame}[label={sec:orgheadline1}]{Webpack}
\begin{itemize}
\item Bundlers
\item Packs web related things (js, css, html) to one bundle file
\item Similar to browserify
\item More popular in React ecosystem
\item Webpack supports ES6 modules, commonjs modules and amd modules.
\end{itemize}
\end{frame}

\begin{frame}[label={sec:orgheadline2}]{CommonJS modules}
\begin{itemize}
\item We will focus on this as it is the same as in node.
\item Each module exports either a single item or an object
\end{itemize}
\end{frame}

\begin{frame}[fragile,label={sec:orgheadline3}]{Single item exported}
 \begin{itemize}
\item \texttt{module.exports = foo} in foo.js
\item Usage: \texttt{foo = require('./foo')} in bar.js in same dir.
\end{itemize}
\end{frame}

\begin{frame}[fragile,label={sec:orgheadline4}]{Multiple items exported and required}
 \begin{itemize}
\item \texttt{module.exports.bar = bar; module.exports.baz = baz} in foo.js
\item Usage: \texttt{foo = require('./foo'); foo.bar()} in second.js
\end{itemize}
\end{frame}

\begin{frame}[fragile,label={sec:orgheadline5}]{Alternative syntax}
 \begin{itemize}
\item \texttt{module.exports = \{bar: bar, baz: baz\}};

\item ES6 shortcut: \texttt{module.exports = \{bar, baz\}};

\item For requiring ES6 shortcut: \texttt{\{bar, baz\} = require('./foo')} (destructuring)
\end{itemize}
\end{frame}

\begin{frame}[fragile,label={sec:orgheadline6}]{React components}
 \begin{itemize}
\item Need a way to transform JSX to JS
\item webpack has loaders
\item either jsx-loader or babel-loader
\item babel-loader is more comprehensive
\item \texttt{npm install babel-loader -{}-save}
\end{itemize}
\end{frame}

\begin{frame}[fragile,label={sec:orgheadline7}]{webpack.config.js}
 \begin{itemize}
\item Need a config file to run webpack
\item This specifies the entry and output files, and the loaders to use.
\end{itemize}

\begin{verbatim}
module.exports = {
  entry: './component.js',
  output: {path: 'build', filename: 'bundle.js'},
  module: {
    loaders: [
      { test: /\.js$/, exclude: /node_modules/,
        loader: "babel-loader"}
    ]
  }

}
\end{verbatim}

\begin{itemize}
\item We can also require react, underscore etc and they will be bundled as well
if they are installed as node$\backslash$\(_{\text{modules}}\).
\end{itemize}
\end{frame}

\begin{frame}[fragile,label={sec:orgheadline8}]{Production Settings}
 \begin{itemize}
\item Run with \texttt{webpack -p}
\item File name is fixed by default. Append \texttt{[chunkname]}
\end{itemize}
\begin{verbatim}
module.exports = {
  entry: './component.js',
  output: {path: 'build',
           filename: 'bundle.[chunkhash].js'},
  module: {
    loaders: [
      { test: /\.js$/, exclude: /node_modules/,
        loader: "babel-loader"}
    ]
  }
}
\end{verbatim}
\end{frame}

\begin{frame}[fragile,label={sec:orgheadline9}]{Retrieving the output hash}
 \begin{itemize}
\item Run with \texttt{-j} flag to get a json summary.
\item Pipe that through \texttt{jq} to get a specific data in the json output
\end{itemize}

\begin{verbatim}
webpack -j | jq ".assetsByChunkName.main"
\end{verbatim}
\begin{itemize}
\item jq is at \url{http://stedolan.github.io/jq/}
\end{itemize}
\end{frame}

\begin{frame}[fragile,label={sec:orgheadline10}]{Production and Dev in a Single File}
 \begin{verbatim}
module.exports = {
  entry: './component.js',
  output: {path: 'build',
           filename: (
             process.env.PROD ?
               'bundle.[chunkhash].js'
               : 'bundle')},
  module: {
    loaders: [
      { test: /\.js$/, exclude: /node_modules/,
        loader: "babel-loader"}
    ]
  }
}
\end{verbatim}
\end{frame}

\begin{frame}[fragile,label={sec:orgheadline11}]{Using this webpack config for dev and prod}
 \begin{itemize}
\item For dev: run \texttt{webpack}
\item For prod: run \texttt{PROD=true webpack -p | jq ".assetsByChunkName.main" >
  bundle\_hash}
\item In your HTML, somehow embed the contents of \texttt{bundle\_hash}
\end{itemize}
\end{frame}

\begin{frame}[fragile,label={sec:orgheadline12}]{Further Tips for Development}
 \begin{itemize}
\item Watch mode: -w flag
\item Development static server

\item Install and run webpack-dev-server.

\item Point to the following for websocket connection:

\item \texttt{<script src="http://localhost:8080/webpack-dev-server.js"></script>}

\item The bundle will be generated at \texttt{http://localhost:8080/bundle.js}

\item Point to this url in your HTML.

\item Make changes to your JS files and ensure that the page is refreshed
automatically.
\end{itemize}
\end{frame}

\begin{frame}[fragile,label={sec:orgheadline13}]{Exercise:}
 \begin{itemize}
\item Study the solution at \texttt{/solutions/01\_hello\_world\_webpack}
\item Go through the sample webpack example to make sure you understand how
webpack works, and how its auto reloading dev server works.
\item Make some changes to the hello world application and make sure that auto
reload works as advertised.
\item Componentize the todo application so that each component lives in a separate
file and compile it to a single bundle file using webpack.
\end{itemize}
\end{frame}
\end{document}
