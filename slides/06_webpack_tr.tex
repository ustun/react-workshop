% Created 2015-06-13 Sat 08:24
\documentclass[presentation]{beamer}
\usepackage[utf8]{inputenc}
\usepackage[T1]{fontenc}
\usepackage{fixltx2e}
\usepackage{graphicx}
\usepackage{longtable}
\usepackage{float}
\usepackage{wrapfig}
\usepackage{rotating}
\usepackage[normalem]{ulem}
\usepackage{amsmath}
\usepackage{textcomp}
\usepackage{marvosym}
\usepackage{wasysym}
\usepackage{amssymb}
\usepackage{hyperref}
\tolerance=1000
\usetheme{Rochester}
\author{Ustun Ozgur}
\date{2015-06-13 Cumartesi}
\title{Webpack and React}
\hypersetup{
  pdfkeywords={},
  pdfsubject={},
  pdfcreator={Emacs 24.4.1 (Org mode 8.2.10)}}
\begin{document}

\maketitle

\begin{frame}[label=sec-1]{Webpack}
\begin{itemize}
\item Bundler (birleştirici)
\item pack (birleştirmek)
\item weble ilgili şeyleri (js, css, html) tek bir dosyaya birleştirme
\item browserify benzeri
\item React ekosisteminde daha popüler
\item ES6 modüllerini, commonjs modüllerini ve amd modüllerini destekler.
\end{itemize}
\end{frame}

\begin{frame}[label=sec-2]{CommonJS modülleri}
\begin{itemize}
\item Node.js'teki modüller
\item Her modül tek bir eleman ya da nesne export eder (dışa aktarır)
\end{itemize}
\end{frame}

\begin{frame}[fragile,label=sec-3]{Tek eleman dışa aktarma item exported}
 \begin{itemize}
\item foo.js dosyasında: \verb~module.exports = foo~
\item Kullanım: \texttt{foo = require('./foo')} aynı klasördeki bar.js'te
\end{itemize}
\end{frame}

\begin{frame}[fragile,label=sec-4]{Birden fazla eleman dışa aktarma ve içe aktarma}
 \begin{itemize}
\item \verb~module.exports.bar = bar; modüle.exports.baz = baz~ foo.js
\item Kullanım: second.js'te \verb~foo = require('./foo'); foo.bar()~
\end{itemize}
\end{frame}

\begin{frame}[fragile,label=sec-5]{Alternatif sözdizim}
 \begin{itemize}
\item \verb~module.exports = {bar: bar, baz: baz}~;

\item Dışa aktarmak için ES6 kısayolu: \verb~module.exports = {bar, baz}~;

\item İçe aktarmak için ES6 kısayolu: \verb~{bar, baz} = require('./foo')~
  (destructuring) (parçalama)
\end{itemize}
\end{frame}

\begin{frame}[fragile,label=sec-6]{React bileşenleri}
 \begin{itemize}
\item JSX'ten JS oluşturmak için dönüştürücü modül
\item webpack'ın loader'ları var
\item jsx-loader ya da babel-loader
\item babel-loader daha iyi.
\item \texttt{npm install babel-loader -{}-save}
\end{itemize}
\end{frame}

\begin{frame}[fragile,label=sec-7]{webpack.config.js}
 \begin{itemize}
\item webpack çalıştırmak için bir config dosyasına ihtiyaç var
\item Bu girdi ve çıktıyı belirler ve kullanılacak loaderları gösterir
\end{itemize}

\begin{verbatim}
module.exports = {
  entry: './component.js',
  output: {path: 'build', filename: 'bundle.js'},
  module: {
    loaders: [
      { test: /\.js$/, exclude: /node_modules/,
        loader: "babel-loader"}
    ]
  }

}
\end{verbatim}

\begin{itemize}
\item react, underscore gibi kütüphaneleri kullanabiliriz ve bunlar da
nodemodules içine kopyalanacaktır.
\end{itemize}
\end{frame}

\begin{frame}[fragile,label=sec-8]{Production Ayarları}
 \begin{itemize}
\item \texttt{webpack -p} ile çalıştırın.
\item Dosya adı varsayılan olarak sabit. Değiştirmek için \texttt{[chunkname]}
\end{itemize}
\begin{verbatim}
module.exports = {
  entry: './component.js',
  output: {path: 'build',
           filename: 'bundle.[chunkhash].js'},
  module: {
    loaders: [
      { test: /\.js$/, exclude: /node_modules/,
        loader: "babel-loader"}
    ]
  }
}
\end{verbatim}
\end{frame}

\begin{frame}[fragile,label=sec-9]{Çıktı dosyasının hash'ini almak}
 \begin{itemize}
\item \texttt{-j} bayrağı ile bir json özet al
\item Bunu \texttt{jq} üzerinden geçirerek json çıktısındaki belli bir veriyi al
\end{itemize}

\begin{verbatim}
webpack -j | jq ".assetsByChunkName.main"
\end{verbatim}
\begin{itemize}
\item jq şuradan indirilebilir: \url{http://stedolan.github.io/jq/}
\end{itemize}
\end{frame}

\begin{frame}[fragile,label=sec-10]{Production ve Dev Ayarları Tek Bir Dosyada}
 \begin{verbatim}
module.exports = {
  entry: './component.js',
  output: {path: 'build',
           filename: (
             process.env.PROD ?
               'bundle.[chunkhash].js'
               : 'bundle')},
  module: {
    loaders: [
      { test: /\.js$/, exclude: /node_modules/,
        loader: "babel-loader"}
    ]
  }
}
\end{verbatim}
\end{frame}

\begin{frame}[fragile,label=sec-11]{webpack'i dev ve prod için kullanmak}
 \begin{itemize}
\item dev için: \texttt{webpack}
\item prod için: \verb~PROD=true webpack -p | jq ".assetsByChunkName.main" >   bundle_hash~
\item HTM:'de bu \texttt{bundle\_hash} içeriğini gömün
\end{itemize}
\end{frame}

\begin{frame}[fragile,label=sec-12]{Geliştirme için İpuçları}
 \begin{itemize}
\item İzleme modu: -w bayrağı
\item Geliştirme için statik sunucu

\item webpack-dev-server yükleyin ve çalıştırın.

\item Şu websocket bağlantısına link verin:

\item \verb~<script src="http://localhost:8080/webpack-dev-server.js"></script>~

\item Çıktı şu adreste geçerli olacaktır \texttt{http://localhost:8080/bundle.js}

\item HTML'de bu adrese referans verin.

\item JS dosyanızda değişiklik yapın ve dosyanın otomatik olarak güncellendiğini görün.
\end{itemize}
\end{frame}

\begin{frame}[fragile,label=sec-13]{Alıştırma:}
 \begin{itemize}
\item Şuradaki örneği inceleyin: \texttt{/solutions/01\_hello\_world\_webpack}
\item Örnek üzerinden giderek webpack'ı çalıştırın ve otomatik reload yapan dev
server'i çalıştırın.
\item Hello world uygulamasında bazı değişiklikler yapın, otomatik reload'ın
çalıştığından emin olun.
\item todo uygulamasını bileşenlere ayırın, bu bileşenleri farklı dosyalara atın,
ve tek bir dosyada bu bileşenleri require edin.
\item webpack ıle bu dosyaları tek bir dosyada birleştirin.
\end{itemize}
\end{frame}
% Emacs 24.4.1 (Org mode 8.2.10)
\end{document}
