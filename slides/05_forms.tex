% Created 2015-05-26 Tue 06:49
\documentclass[presentation]{beamer}
\usepackage[utf8]{inputenc}
\usepackage[T1]{fontenc}
\usepackage{fixltx2e}
\usepackage{graphicx}
\usepackage{longtable}
\usepackage{float}
\usepackage{wrapfig}
\usepackage{rotating}
\usepackage[normalem]{ulem}
\usepackage{amsmath}
\usepackage{textcomp}
\usepackage{marvosym}
\usepackage{wasysym}
\usepackage{amssymb}
\usepackage{hyperref}
\tolerance=1000
\usetheme{default}
\usecolortheme{spruce}
\author{Ustun Ozgur}
\date{2015-05-26 Tue}
\title{Forms with React}
\hypersetup{
  pdfkeywords={},
  pdfsubject={},
  pdfcreator={Emacs 24.4.1 (Org mode 8.2.10)}}
\begin{document}

\maketitle

\begin{frame}[label=sec-1]{Forms}
\begin{itemize}
\item Uncontrolled
\item Controlled
\end{itemize}
\end{frame}

\begin{frame}[fragile,label=sec-2]{Uncontrolled}
 \begin{itemize}
\item Discouraged
\item Native DOM elements are the source of truth, not sate
\item Values read from events or DOM elements through refs
\item ref: unique identifier for a child
\item \texttt{this.refs} an object like \texttt{props}, \texttt{state} that is an object of refs
\end{itemize}
\end{frame}

\begin{frame}[fragile,label=sec-3]{Refs}
 \begin{itemize}
\item Example
\end{itemize}
\begin{verbatim}
<div><span ref="inner">This is the inner text</span></div>
\end{verbatim}

\begin{itemize}
\item The inner span can  be referred as \texttt{this.refs.inner}

\item To retrieve actual DOM node, use:
\begin{itemize}
\item \texttt{React.findDOMNode}: preferred
\item \texttt{this.refs.inner.getDOMNode}: deprecated
\end{itemize}

\item To retrieve the value, use value after getting the DOM node
\item \texttt{this.refs.inner.getDOMNode().value}
\end{itemize}
\end{frame}


\begin{frame}[fragile,label=sec-4]{Example: Form with Uncontrolled Inputs}
 \begin{verbatim}
<form onSubmit={this.onSubmit}>

<input ref={name} placeholder="What is your name"/>
<input ref={surname} placeholder="What is your surname?"/>
<button type="submit">Submit</button>

</form>
\end{verbatim}
\end{frame}

\begin{frame}[fragile,label=sec-5]{Example (cont'd)}
 \begin{verbatim}
var MyForm = React.createClass({
render: function () {

  return <form onSubmit={this.onSubmit}>
    <input ref="name" placeholder="What is your name"/>
    <input ref="surname"
     placeholder="What is your surname?"/>
    <textarea ref="comment" placeholder="Comment"/>
    <label><input ref="fromDenmark"
            type="checkbox"/>Are you from Denmark?</label>
    <label>Gender:
    <select ref="gender">
    <option>Male</option>
    <option>Female</option>
    </select></label>
    <button type="submit">Submit</button>

    </form>
\end{verbatim}
\end{frame}


\begin{frame}[fragile,label=sec-6]{Example (cont'd)}
 \begin{verbatim}
onSubmit: function (e) {
  e.preventDefault();
  var name = React.findDOMNode(this.refs.name).value;
  var surname = React
        .findDOMNode(this.refs.surname).value;
  var comment = React
        .findDOMNode(this.refs.comment).value;
  var fromDenmark = React
        .findDOMNode(this.refs.fromDenmark).checked;
  var gender = React.findDOMNode(this.refs.gender).value;

  console.log("Form submitted with",
              name, surname, comment,
              fromDenmark, gender);

}});

React.render(<MyForm/>, document.body);
\end{verbatim}
\end{frame}

\begin{frame}[label=sec-7]{Observations}
\begin{itemize}
\item For checkboxes, we use checked attribute, not value.
\item textarea does not have children content unlike HTML. Its value can be
retrieved/set from its value attribute.
\item Forms have an onSubmit event handler, however this by default calls the
default form handler as well. We need to preventDefault.
\item Events are synthetic events that wrap and normalize native DOM events.
\end{itemize}
\end{frame}

\begin{frame}[label=sec-8]{Controlled Forms}
\begin{itemize}
\item Idiomatic way
\item Each input component accepts a value or checked property
\item Each input component exposes onChange
\item Set the value from the state
\item Update the state in onChange
\end{itemize}
\end{frame}

\begin{frame}[fragile,label=sec-9]{Example: Controlled Input (initial state)}
 \begin{verbatim}
var MyForm = React.createClass({

  getInitialState: function () {
    return {name: ''};
  },
  render: function () {
    return <form onSubmit={this.onSubmit}>
      <input value={this.state.name} placeholder="What is your name?"/>
      </form>
  }
})
\end{verbatim}
\end{frame}

\begin{frame}[fragile,label=sec-10]{Example (cont'd) (the event handlers)}
 \begin{verbatim}
var MyForm = React.createClass({

  getInitialState: function () {
    return {name: ''};
  },
  changeName: function (e) {
    this.setState({name: e.target.value});
  },
  render: function () {
    return <form onSubmit={this.onSubmit}>
      <input onChange={this.changeName}
       value={this.state.name}
       placeholder="What is your name?"/>
      </form>}})
\end{verbatim}
\end{frame}

\begin{frame}[fragile,label=sec-11]{Example (the submit handler)}
 \begin{verbatim}
var MyForm = React.createClass({
  onSubmit: function () {
    console.log("the form values are", this.state.name); },
  getInitialState: function () {
    return {name: ''};  },
  changeName: function (e) {
    this.setState({name: e.target.value});  },
  render: function () {
    return <form onSubmit={this.onSubmit}>
      <input onChange={this.changeName}
       value={this.state.name}
       placeholder="What is your name?"/>
      </form>  }})
\end{verbatim}
\end{frame}


\begin{frame}[label=sec-12]{Other Events Related to Forms}
\begin{itemize}
\item onBlur
\item onFocus
\end{itemize}
\end{frame}

\begin{frame}[fragile,label=sec-13]{A Debugging Trick}
 \begin{itemize}
\item A nice trick in debugging form is to output state visually
\item \texttt{JSON.stringify(this.state, null, 4)} yields a properly indented version of
state.
\end{itemize}
\end{frame}

\begin{frame}[fragile,label=sec-14]{Example:}
 \begin{verbatim}
var MyForm = React.createClass({
  onSubmit: function () {
    console.log("the form values are", this.state.name); },
  getInitialState: function () {
    return {name: ''};  },
  changeName: function (e) {
    this.setState({name: e.target.value});  },
  render: function () {
    return <form onSubmit={this.onSubmit}>
      <input onChange={this.changeName}
       value={this.state.name}
       placeholder="What is your name?"/>
      <pre>{JSON.stringify(this.state, null, 4)}</pre>
      </form>  }})
\end{verbatim}


DEMO
\end{frame}

\begin{frame}[fragile,label=sec-15]{Exercise 1/2:}
 \begin{itemize}
\item Add another form input for credit card where the user can only enter numbers.
\end{itemize}

Hint: You can use \texttt{/\textasciicircum{}\textbackslash{}d+\$/.test(foo)} to test whether the variable `foo` consists of only numbers.

\begin{itemize}
\item Can you delete the card number after entering a few digits? If not, fix the
bug. Hint: either change the regex or find some other means.

\item As the user types in the name field, greet them with a gender prefix. The
male names are: \texttt{["John", "George"]}. The female names are \texttt{["Jane",
  "Mary"]}.

\item If gender cannot be determined from the name, greet with just the name. If
there is no name yet, do not greet.
\end{itemize}
\end{frame}

\begin{frame}[label=sec-16]{Exercise 2/2:}
\begin{itemize}
\item As the user passes from the name field to the card field, validate the name
such that it is at least 3 letters. If the name is 2 letters, show a
warning. Note that you should not show the warning initially.
\end{itemize}

Hint: Think about state variables to keep track of. Should the variable that
determines whether the input is valid or not be stored in state? Think about
pros and cons.

\begin{itemize}
\item Add validation on submit. The name should be at least 3 letters, the card
should be at least 3 digits. Should we be storing the validation of the form
in state? Think about it.
\end{itemize}
\end{frame}
% Emacs 24.4.1 (Org mode 8.2.10)
\end{document}
