% Created 2015-05-26 Tue 12:18
\documentclass[presentation]{beamer}
\usepackage[utf8]{inputenc}
\usepackage[T1]{fontenc}
\usepackage{fixltx2e}
\usepackage{graphicx}
\usepackage{longtable}
\usepackage{float}
\usepackage{wrapfig}
\usepackage{rotating}
\usepackage[normalem]{ulem}
\usepackage{amsmath}
\usepackage{textcomp}
\usepackage{marvosym}
\usepackage{wasysym}
\usepackage{amssymb}
\usepackage{hyperref}
\tolerance=1000
\usetheme{default}
\usecolortheme{spruce}
\author{Ustun Ozgur}
\date{2015-05-26 Tue}
\title{Flux}
\hypersetup{
  pdfkeywords={},
  pdfsubject={},
  pdfcreator={Emacs 24.4.1 (Org mode 8.2.10)}}
\begin{document}

\maketitle


\begin{frame}[label=sec-1]{Flux}
\begin{itemize}
\item open sourced by Facebook
\item not a library, a technique like MVC
\item suitable for bigger applications
\end{itemize}
\end{frame}

\begin{frame}[label=sec-2]{When is Flux a good idea?}
\begin{itemize}
\item Too deep level in React hiearchy
\item Need to pass props to the bottom too much
\item Two components not having a single parent, but want to sync state
\end{itemize}
\end{frame}


\begin{frame}[label=sec-3]{Stores}
\begin{itemize}
\item Enclose the domain logic
\item Act as event emitters
\item Emits a change event whenever it changes
\item Simple version, not Flux: No actions
\item React component intercepts the events, forwards them to store
\item React component subscribe to store events
\item React component syncs its state with store
\end{itemize}
\end{frame}

\begin{frame}[label=sec-4]{Actions}
\begin{itemize}
\item As apps get big, react component should not call store directly
\item Instead a level of indirection
\item React components intercept events
\item Forward them to dispatcher as events
\end{itemize}
\end{frame}

\begin{frame}[label=sec-5]{Dispatcher}
\begin{itemize}
\item Dispatcher broadcasts events to all stores
\item Stores decide which events they are interested in
\item They decide accordingly and emit a change event
\item React components sync accordingly
\end{itemize}
\end{frame}

\begin{frame}[label=sec-6]{Flux Architecture (1/3)}
\includegraphics[width=.9\linewidth]{./flux1.png}
\end{frame}



\begin{frame}[label=sec-7]{Flux Architecture (2/3)}
\includegraphics[width=.9\linewidth]{./flux2.png}
\end{frame}
\begin{frame}[label=sec-8]{Flux Architecture (3/3)}
\includegraphics[width=.9\linewidth]{./flux3.png}
\end{frame}
\begin{frame}[label=sec-9]{Detail}
\begin{itemize}
\item \emph{Demo}
\end{itemize}
\end{frame}
% Emacs 24.4.1 (Org mode 8.2.10)
\end{document}
