% Created 2015-06-13 Sat 08:29
\documentclass[presentation]{beamer}
\usepackage[utf8]{inputenc}
\usepackage[T1]{fontenc}
\usepackage{fixltx2e}
\usepackage{graphicx}
\usepackage{longtable}
\usepackage{float}
\usepackage{wrapfig}
\usepackage{rotating}
\usepackage[normalem]{ulem}
\usepackage{amsmath}
\usepackage{textcomp}
\usepackage{marvosym}
\usepackage{wasysym}
\usepackage{amssymb}
\usepackage{hyperref}
\tolerance=1000
\usetheme{Rochester}
\author{Üstün Özgür}
\date{2015-06-13 Cumartesi}
\title{React ile statik sayfa görüntüleme}
\hypersetup{
  pdfkeywords={},
  pdfsubject={},
  pdfcreator={Emacs 24.4.1 (Org mode 8.2.10)}}
\begin{document}

\maketitle

\begin{frame}[label=sec-1]{React ile statik sayfa görüntüleme}
\begin{block}{\emph{render} metodu}
\begin{itemize}
\item render'da bir JSX ifadesi döneriz
\item Sıradan HTML elemanları ya da birleşik (kompozit) React bileşenleri
\end{itemize}
\end{block}
\end{frame}

\begin{frame}[fragile,label=sec-2]{JSX Hakkında}
 \begin{itemize}
\item Bir HTML ya da XML sürümü değil
\item Normal JavaScript'e dönüştürülür
\item \verb~<div ad="foo" soyad="bar">içerik</div>~
\item \texttt{React.createElement('div', (\{ad: "foo", soyad: "bar"\}, "icerik")}
\item Özellikler \emph{props} adı verilen bir nesnede (object) toplanır
\item Çocuklar sonraki argümanlar olarak geçirilir
\item \texttt{fonksiyon(props, çocuklar)}
\item İç içe gecebilir:: Birleşik (kompozit) bileşenler
\end{itemize}
\end{frame}

\begin{frame}[fragile,label=sec-3]{Daha karmaşık bir örnek}
 \begin{block}{JSX}
\begin{verbatim}
<div ad="foo">
      <span>Bar</span>
      <span>Baz</span>
</div>
\end{verbatim}
\end{block}
\begin{block}{JS}
\begin{verbatim}
React.createElement('div', {ad: 'foo'},
   React.createElement('span', {}, "Bar"),
   React.createElement('span', {}, "Baz"))
\end{verbatim}
\end{block}
\end{frame}

\begin{frame}[label=sec-4]{JSX: 5 Dakika Verin}
\begin{itemize}
\item JS içinde HTML mi?
\item HTML içinde JS mi?
\item Aslında JS ve HTML karışımı değil, sadece JS
\end{itemize}
\end{frame}

\begin{frame}[fragile,label=sec-5]{Ana Özellikler}
 \begin{itemize}
\item Sadece tek bir eleman dön
\begin{itemize}
\item Bir fonksiyonun tek bir çıktısı olması gibi
\end{itemize}
\item Bütün elemanlar kapanmalı ya da kendiliğinden kapanır olmalı
\begin{itemize}
\item \texttt{<div></div>}, \texttt{<img/>}, \texttt{<input/>}
\end{itemize}
\item JSX ifadeleri koymak için süslü parantezler
\begin{itemize}
\item \texttt{<div>\{2 + 3\}</div>} Geçerli JSX
\end{itemize}
\end{itemize}
\end{frame}

\begin{frame}[fragile,label=sec-6]{Ana Özellikler}
 \begin{verbatim}
var icTaraf = <div>Merhaba</div>

var disTaraf = <div>{icTaraf} {icTaraf}</div>
\end{verbatim}
\begin{itemize}
\item String birleşimi değil. \texttt{icTaraf} herhangi bir değişken gibi
\end{itemize}
\begin{verbatim}
var icerikler = [];

icerikler.push(icTaraf);
icerikler.push(icTaraf);

var disTaraf = <div>{icerikler}</div>
\end{verbatim}
\end{frame}

\begin{frame}[fragile,label=sec-7]{Ana Özellikler}
 \begin{verbatim}
var isimler = ["Ali", "Ahmet", "Mehmet"];

var icEleman = isimler.map(function (ad) {
    return <div>Merhaba {ad}</div>;
});

var disEleman = <div>{icEleman}</div>
\end{verbatim}
\end{frame}

\begin{frame}[fragile,label=sec-8]{Ana Özellikler}
 \begin{verbatim}
var icerik;

if (this.props.renk == "kirmizi") {
   icerik = <div>Renk kırmızı.</div>
}


dis = <div>{icerik}</div>
\end{verbatim}
\end{frame}

\begin{frame}[fragile,label=sec-9]{Ana Özellik}
 \begin{itemize}
\item If ifadeleri statement'tır, expression değildir
\item Ama üç terimliler expression:
\end{itemize}
\begin{verbatim}
a == 3 ? "a 3" : "a 3 değil"
\end{verbatim}

\begin{block}{Örnek}
\begin{verbatim}
outer = <div>{this.props.renk == "kırmızı" ?
                <div>Renk kırmızı</div>:  null}</div>
\end{verbatim}

\begin{verbatim}
outer = <div>{this.props.color == "kırmızı" &&
               <div>Renk kırmızı</div>}</div>
\end{verbatim}

\begin{itemize}
\item Okumayı zorlaştırabilir
\end{itemize}
\end{block}
\end{frame}

\begin{frame}[fragile,label=sec-10]{JSX içinde CSS}
 \begin{itemize}
\item JSX aslında arka planda sadece JS
\end{itemize}
\begin{block}{class}
\begin{itemize}
\item \texttt{class} kelimesi JS içinde rezerve (anahtar kelime)
\item Bunun yerine \texttt{className} kullanılır
\item \verb~<div class="foo bar">~ suna dönüşür:  \verb~<div className="foo bar">~
\end{itemize}
\end{block}
\begin{block}{Stiller}
\begin{itemize}
\item HTML'deki stiller string'dir. CSS'te, JS'teki nesnelere benzer bir sözdizim
kullanılır.
\item JSX'te stiller gerçek JS nesneleridir.
\end{itemize}
\end{block}
\end{frame}

\begin{frame}[fragile,label=sec-11]{JSX içinde CSS (devam)}
 \verb~<div style="color: red; background-color: yellow">~
\begin{itemize}
\item String bir JS nesnesine dönüştürülür
\end{itemize}
\texttt{\{color: 'red', backgroundColor: 'yellow'\}}
\verb~<div style={{color: 'red', backgroundColor: 'yellow'}}~
\begin{itemize}
\item Dikkat: İki tane \texttt{\{}: Birincisi JSX için, ikincisi JS nesnesi için
\item Tire yok: Bunun yerine sonraki harfi büyük harfe çevir
\texttt{background-color} yerine \texttt{backgroundColor}
\end{itemize}
\end{frame}

\begin{frame}[fragile,label=sec-12]{JSX içinde CSS (devam)}
 \begin{itemize}
\item JSX'te sayılar: CSS'te çoğu birim px'tır, bu yüzden otomatik olarak eklenir
\end{itemize}
\begin{verbatim}
fontSize:'12px'
\end{verbatim}
ifadesi şu şekilde yazılabilir
\begin{verbatim}
fontSize: 12
\end{verbatim}
\end{frame}

\begin{frame}[fragile,label=sec-13]{JSX içinde CSS (devam)}
 \begin{itemize}
\item Her zaman beklendiği gibi çalışmaz. Örneğin lineHeight için ana birim 'px'
değil 'em'dir
\end{itemize}
\begin{verbatim}
lineHeight: 18 ifadesi lineHeight: "18em" anlamına gelir
\end{verbatim}

Şunu yazın:
\begin{verbatim}
lineHeight: '18px'
\end{verbatim}
\end{frame}

\begin{frame}[label=sec-14]{Var olan bir HTML'i React'e çevirmek}
\begin{enumerate}
\item App adında bir React bileşeni oluşturun. Hazır HTML'i alın ve render
metodunda bunu dondurun.

\item Gerekli CSS değişikliklerini yapın. class: className. stiller: nesne

\item Bütün etiketlerin (taglerin) doğru kapandığından emin olun.

\item App.js dosyasını HTML'den çağırın, HTML'e React bileşeninizin
yerleştirileceği ana HTML bileşenini koyun ve React bileşenini `React.render`
ile yerleştirin (mount). (Doğrudan body'ye de yerlestirilebilir.)

\item 2. ve 3. adımlar için dönüştürücüyü (transpiler) izleme (watch) modunda
çalıştırın. Hata vermeyene kadar değişiklikleri yapın.
\end{enumerate}
\end{frame}

\begin{frame}[fragile,label=sec-15]{JSX'i JavaScript'e Çevirmek İçin Araçlar}
 \begin{itemize}
\item Dönüştürücü (Transpilation)
\item \alert{react-tools}'u şununla yükleyin: \texttt{npm install -g react-tools}
\item İki mod: Tek dosya girişi  ya da klasör girişi
\end{itemize}
\end{frame}

\begin{frame}[fragile,label=sec-16]{Basit Bir Örnek}
 \begin{verbatim}
var MerhabaDunya = React.createClass({
    render: function () {
            return <div>Merhaba Dunya</div>
    }
})
\end{verbatim}

HTML kısmında ana bileşen için bir içerici (container)
\begin{verbatim}
<div id="app"></div>
\end{verbatim}
Yerleştirmek (monte) için:
\begin{verbatim}
React.render(<MerhabaDunya/>, document.getElementById('app'))
\end{verbatim}
\end{frame}

\begin{frame}[fragile,label=sec-17]{Alıştırma 1: React ile Başlangıç: 5-10 dakika}
 İpucu: Makefile dosyasına bakabilirsiniz.

a - Depoyu klonlayın.

b - \texttt{examples/01\_hello\_world} klasörüne gidin.

c - react.js'i indirin, react-tools yükleyin, daha sonra uygulamayı jsx
kullanarak derleyin, ve tarayıcıda "Merhaba Dünya"'yı görüntüleyin.

d - Uygulamayı "Merhaba Istanbul!" diye değiştirin.

e - Aynısını şu klasör için tekrarlayin \texttt{examples/01\_hello\_world\_watch} ama bu
sefer jsx'i izleme modunda çalıştırın.

f - Bonus: React.render'i doğrudan HTML içindeki bir script içine nasıl
yerleştirirdiniz. JSX'i elle bir React.createElement çağrısına dönüştürmeniz gerekir.

g - Soru: Niye script'i body'nin sonuna koyuyoruz?
\end{frame}

\begin{frame}[fragile,label=sec-18]{Daha Karmaşık bir örnek: ToDo Uygulaması}
 \begin{itemize}
\item \texttt{examples/02\_todo\_props/mockup.html} alıp React'e çevirin.
\item Çevrimiçi HTML-JSX dönüştürücü :
\url{https://facebook.github.io/react/html-jsx.html}
\item Komut satırı versiyonu: \url{https://www.npmjs.com/package/htmltojsx}
\item Sonucu \texttt{src/todo.js} klasörüne kopyalayın.
\item jsx'i klasör modunda çalıştırın: \texttt{jsx src build}
\end{itemize}
\end{frame}

\begin{frame}[fragile,label=sec-19]{Alıştırma 2:}
 a - \texttt{02\_convert\_mockup/mockup.html}'da verilen HTML'i React'e çevirin. Dikkat:
Burada iki farklı ana kısım var, o yüzden iki ayrı kök eleman (node) olacak.

b - ToDo uygulamasını küçük bileşenlere parçalayın. Şu adlarda bileşenler
olusturun: SearchBar, Todos, TodoItem, Footer.
\end{frame}
% Emacs 24.4.1 (Org mode 8.2.10)
\end{document}
